\documentclass[12pt,a4paper]{jsarticle}
\AtBeginDvi{\special{pdf:mapfile ptex-ms.map}}

\usepackage{amsmath,amssymb}
\usepackage{graphicx}
\usepackage[dvipdfmx]{xcolor}
\usepackage{float}
\usepackage{tikz}
\usepackage{circuitikz}
\usepackage{siunitx}
\usepackage{at}
\usepackage{comment}
\usepackage[hang,small,bf]{caption}
\usepackage[subrefformat=parens]{subcaption}
\usepackage{ascmac}
\usepackage{enumerate}
\usepackage{rotating}
\usepackage{otf}
\usepackage{textcomp}
\usepackage{mathcomp}
\usepackage{here}
\usepackage{fancyvrb}
\usepackage{booktabs}
\usepackage{circledsteps}
\usepackage[top=10truemm,bottom=20truemm]{geometry}
\usepackage{spverbatim}
\usepackage{multirow}
\usepackage{url}
\usepackage{listings, jvlisting}
\lstset{
basicstyle={	tfamily},
identifierstyle={\small},
commentstyle={\smallitshape},
keywordstyle={\bfseries},
ndkeywordstyle={\small},
stringstyle={\normalsize\ttfamily},
frame={tb},
breaklines=true,
showstringspaces=false,
columns=fixed,
basewidth=0.5em,
numberstyle={\normalsize},
numbers=left,
captionpos=b
numbersep=1zw,
lineskip=-0.3ex
}

\captionsetup{compatibility=false}

\usetikzlibrary{intersections,calc,arrows.meta}

\numberwithin{equation}{section}
\numberwithin{figure}{section}
\numberwithin{table}{section}
\usepackage{chngcntr}
\AtBeginDocument{\counterwithin{lstlisting}{section}}
\renewcommand{\lstlistingname}{ソースコード}

\newcommand{\myscope}[2] % #1 = name , #2 = rotation angle
{draw[thick,rotate=#2] (#1) circle (16pt)
 (#1) ++(-0.35,-0.1) -- ++(0.3,0.3) --++(0,-0.3)-- ++(0.3,0.3) --++(0,-0.3);
}

\title{「蜂蜜パイ」の熊のまさきちのお話が表象しているものとそれにみる淳平の心情の変化}
\author{4405 井下 颯力}
\date{表象文化I 2024年1月19日}

\begin{document}
\maketitle

\section{はじめに}
本レポートでは、「蜂蜜パイ」の熊のまさきちのお話にみる淳平の心情の変化や、それを裏付ける背景事情について考察していきます。
「蜂蜜パイ」は、2000年2月に初版が出版された村上春樹の短編集「神の子どもたちはみな踊る」
に収録されている小説の1つです。「蜂蜜パイ」には小説家の淳平、新聞記者の高槻、彼らの親友であり高槻の妻であった
小夜子、それら2人の間に生まれた沙羅という4人の登場人物がおり、熊のまさきちのお話(以後、「まさきち」)は作中で淳平が沙羅に対してその場で考え聞かせている物語です。

\section{考察}
先にお伝えしておくと、淳平はまさきち、高槻がとんきちを表象したものになっていると考えられます。

189ページから始まる「まさきち」は、まさきちがどのような熊であるか、どのような関係のなかで暮らしているのかが描写されています。
まさきちはおとなしい性格で、淳平は人見知りであると記されており、共通していると見ることができるでしょう。
その中で、とんきちという熊がまさきちを煙たがっていることが説明されており、これは高槻と小夜子が離婚することになった際に、
淳平が高槻から自身の後釜として小夜子と結婚するように勧められたことに対する回答に悩んでいることの表れであると考えられます。
% この時点では淳平は小夜子と自身が結婚することに懐疑的であり、
また、まさきちが蜂蜜とりの名人であることが示されていますが、これは淳平が多くの女性と付き合い、体を重ねていることを表しているのではないかと考えました。
「蜂蜜パイ」そして「腎臓石」には、キリエをはじめとした多くの女性と淳平が交際し、
別れていることが描写されています。
加えて、その「まさきち」を語り終えたタイミングで小夜子がシューベルトの「鱒」のメロディーをハミングしています。
「鱒」は狡賢い漁師が罠を使って魚を釣るさまを歌ったものでありますが、「女をたぶらかす男には気をつけよ」という忠告のアレゴリーとなっており[1]、
淳平のことを暗に示しているのではないかと考えられます。

次に「まさきち」が語られるのは225ページからになります。
ここではとんきちについて先より詳しく語られており、「それほど頭はよくなかった」と示されており、1年浪人し、それでも他の学部の試験に落ちて文学部に
進学した高槻とも共通していると考えられます。加えて、とんきちは鮭を取ることが上手であることが示されています。
これは高槻が人を捕まえる、つまり友達を作ることが上手であることを表しており、
友達を意味する「鮭」と女性を意味する「蜂蜜」を交換して親友になるというのは、
淳平が密かに思いを抱いていた小夜子が高槻と付き合い、結婚しても尚
親友のままであることを皮肉っているのではないかと考えました。
次に、鮭が突然川から消えてしまい、とんきちが鮭を取られなくなったという話が語られています。
たくさんいた鮭が突然いなくなるということから、阪神淡路大震災を阪神・淡路大震災を表象している
ように見えますが、私はこれは高槻が小夜子と離婚したことを表しているのではないかと考えました。
裏付けとしては、以後に淳平が小夜子と体を交える場面で小夜子が
"「でもあなただけがわからなかった。何もわかっていなかった。鮭が川から消えてしまうまで」"
と発言していることが挙げられます。
その後、とんきちはまさきちの厚意も断り山を降りることを決意し、罠に捕えられるという運び
で一旦は幕を閉じます。
"「まだ思いつかないんだ」"と言っていることからも、小夜子との婚約の決心が未だについて
いないことが伺え、関係を絶つような終わり方にしてしまったと考えられます。
その日の夜、小夜子との情事の後、淳平は紗代子に結婚を申し込むことを決め、「まさきち」
に出口を作らなければならない、救いがなければならないと考え直しました。
そして、蜂蜜パイを焼いて売ることを思いつき、飛ぶように売れて2頭は幸福に暮らしたと締め括りました。
淳平の中で高槻との蜂蜜、つまり女性(ここでは小夜子)の扱いについての議論に折り合いがつけられたために、
先ほどは思いつかなかった「出口」を思いつくことができたと考えられます。

\section{おわりに}
結果として、この「まさきち」には淳平の小夜子を巡る内情の変化が表象されていると考えられます。
最初は小夜子との結婚に懐疑的であった淳平が、最終的には彼女との結婚を決意し、彼女、そして沙羅と
幸せに暮らしていきたいという願望が込められているのではないかと言えるでしょう。


\end{document}
